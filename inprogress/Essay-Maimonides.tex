% use xelatex
\documentclass[12pt]{article}  
\usepackage[left=1in,right=1in,top=1in,bottom=1in]{geometry}
\usepackage{fontspec}
\usepackage{csquotes}
\newfontfamily\myfont[Script=Arabic,Scale=1,Path=/System/Library/Fonts/]{NotoNastaliq.ttc}
\newfontfamily\nyfont[Script=Hebrew,Scale=1,Path=/System/Library/Fonts/]{ArialHB.ttc}
\title{Maimonides on `Athens and Jerusalem'}
\newcommand{\GP}{the \emph{Guide for the Perplexed}\,}
\author{\myfont عماد}
%\date{}							% Activate to display a given date or no date

% -- compile with xelatex -- %
\begin{document}
\maketitle
%\section{}
%\subsection{}
It would be a mistake to think that because Maimonides' attempt to reconcile his faith with his commitments as a philosopher and scientist took place at a time when science looked very different from now, his arguments do not have any bearing on us. In fact, the edifice of putatively `human' (as opposed to `Divine') knowledge that Maimonides was working with --- {\myfont فلسفہ} --- was far more all-embracing and tightly-knit than what we call `Science' today is. 

Under the tutelage of the First Teacher, the Graeco-Arabic tradition had systematized what they considered to be all of human knowledge into an elaborate structure that appeared to bring nearly all the objects of human experience into the purview of \textit{reason}. What is more, the best minds of the day were convinced from scientific principles that all phenomena follow necessarily from the laws of physics, and the religious belief in miracles had to be justified in the face of such a compelling worldview. Impassioned justifications for religious belief were made, then as now, in competition with and in response to a worldview that purports to satisfactorily explain the world without such {\myfont حیلے}. I will use the term \emph{Avicennian necessitarianism} to denote this closed-system view of the universe, which dominated western scientific thinking from Aristotle to Avicenna to Newton to Laplace.

Avicennian necessitarianism was arguably a far more sure-footed challenge to the belief in a personal God who intervenes in the world than that offered by the best science of our day, since Kant has taught us that there remains for ever a curtain between the objects of our knowledge and things in themselves. We are the unfortunate generation who have outlived Laplace's Demon and have learned that even the deterministic universe governed by the physics of Newton entails a certain amount of chaos, rendering moot our vaunted 19th century confidence that this universe of ours could one day be brought into the grasp of our finite intellect.  Even more, we live in a world where we have been forced to contend with the possibility that perhaps God does, indeed, play dice.

Not so Maimonides, who learned from Farabi, channeling Aristotle, that there are exactly ten \emph{Categories} of things, and that virtually all of the objects of our experience can be `made sense of' using a single, self-contained and logically demonstrable, system of science. From Philo of Alexandria he inherited, knowingly or unknowingly, the conviction that God has made the Universe with the same mold from which he fashioned our minds, and a faith in the correspondence of Reason to the objects of our sense-perception. From Plato and perhaps the Pythagoreans he learned that the truths of mathematics are eternal, and that they allow us finite beings to grasp something of the essence of the infinite God. There is a good case to be made that Maimonides actually \emph{worshipped} the Intellect, {\myfont عقل }, and he learned to do this from the one who ``himself taught us the rules of logic, and the means by which arguments can be refuted or confirmed''.

Maimonides' approach to `reconciling' --- an inadequate word for his monumental project, but it will have to suffice here --- his thoroughgoing commitment to reason with his belief in a religion founded on the revelation at Sinai is, at first glance, strange for an avowedly orthodox Rabbi who was widely accepted as the leader of his religious community. 

On the one hand, he believed that human knowledge of the gnostic sort --- {\myfont عرفانی} knowledge --- has its ultimate source in the effusion or {\myfont فیض} from the Divine via the Active Intellect. This conceptual framework allowed him to scientifically explain both Prophecy and Scripture as particularly acute instances of the divine overflow, in which the preparedness of the recipient of revelation determines the degree to which he benefits from the (all-pervading, necessitarian, isotropic, and, strictly speaking, impersonal) {\myfont فیض} of God. 

On the other hand, aware that the literal sense of Scripture could lend itself to the support of beliefs that were incompatible with what he considered to be settled scientific matters --- e.g., the story of the prophet Joshua --- Maimonides was quite willing to interpret the results of Divine Emanation, such as the Tanakh --- allegorically.

In short, he proposed a simple litmus test for the truths of Scripture: does a particular piece of doctrine taught by the plain meaning of Scripture contradict what we know to be true with demonstrative certainty, i.e., by way of {\myfont برہان} ? If so, Scripture must be interpreted in a {\myfont باطنی} fashion, so that its \emph{true} teaching must have been different --- in many cases opposite --- from the apparent one. This position is doubtless what caused consternation among the Rabbis of Provencal who tried to ban the \emph{Guide}, who saw in this dogmatic \emph{rationalism} the seeds for pernicious ideas that would allow believers to discard any part of Scripture that did not correspond to `the science' of their day\footnote{It should be noted that Maimonides was far from being the first Rabbi to take such a radical rationalist position on the relationship between Scripture and scientific Truth; Saadia Gaon (d. 942 A.D.), adopted a similar position in his \emph{kitab al-amanat wa al-i'tiqadat}.}.

On Maimonides' part, this possibility did not worry him too much because he appears, to all accounts, convinced that the revelation at Sinai included not just `apparent truths', the laws that help organize human society in felicitous ways germane to the cultivation of its relationship with God, but also the `eternal truths' of Logic, Mathematics, Physics, and Metaphysics. For Maimonides, the Sages of Israel were custodians of an ancient tradition of \emph{scientific} knowledge, one that the vicissitudes of time had robbed from the Jews and that he was determined to win back for his people from their more recent practitioners. In doing so, he wanted to lead his people into a true recognition of {\nyfont השם}, the one who was at once the God of Aristotle and the God of Moses.

The God that emerges from the \emph{Guide for the Perplexed} after Maimonides' tussle with himself  (apropos Alfred Ivry), is two things at once. He is at the same time (1) Spinoza's God sans extension, i.e., the ontological ground for all of existence, the Formal, Efficient, and Final (but \emph{not} Material) cause of the Universe, and (2) a `God of the gaps', whose direct intervention in the world has to be invoked when --- and \emph{only} when --- science fails to explain some aspects of reality. To be sure, Maimonides' God is quite similar to Spinoza's in the sense that Maimonides appears to be a strong proponent of the principle of Sufficient Reason and is not wont to offer miraculous explanations for phenomena that could plausibly be explained scientifically. He may well have agreed with Spinoza that `Joshua was no astronomer', because his own understanding of the `miracle' of Joshua is quite banal: it was simply ``the longest possible day'' (GP II.35); clearly, he did not think it proper to believe that God somehow lengthened the astronomical day for a battle on earth. However, Maimonides, \emph{pace} Spinoza, seems to believe that Religion, in the last analysis, rests on the idea that there is a God who \emph{intervenes} in the world, however remotely. He believes that Avicennian (and later Spinozist) necessitarianism is not compatible with the God of the Torah, and that God must act `freely', at least in his creation of the world in time if not in the day-to-do workings of the world.

\begin{displayquote}
``If we were to accept the Eternity of the Universe as taught by Aristotle, that everything in the Universe is the result of fixed laws, that Nature does not change, and that there is nothing supernatural, we should necessarily be in opposition to the foundation of our religion, we should disbelieve all miracles and signs, and certainly reject all hopes and fears derived from Scripture, unless the miracles are also explained figuratively.'' \hfill (GP II.25)
\end{displayquote}

A few chapters later, however, we see Maimonides doing exactly this --- he explains one of the miracles figuratively.
\begin{displayquote}
``We must not be misled by the account that the light of the sun stood still certain hours for Joshua, when “he said in the sight of Israel,” etc. (Josh. 10:12); for it is not said there “in the sight of all Israel,” as is said in reference to Moses. So also the miracle of Elijah, at Mount Carmel, was witnessed only by a few people. When I said above that the sun stood still certain hours, I explain the words “ka-jom tamim” to mean “the longest possible day,” because tamim means “perfect,” and indicates that that day appeared to the people at Gibeon as their longest day in the summer.'' \hfill (GP II.35)
\end{displayquote}

Here, Maimonides undermines the veracity of the claim that the sun \emph{literally} `stood still certain hours' in two ways: firstly, by claiming that it was not witnessed by everyone but only by `a few people', and secondly, by explaining it in a non-miraculous way. This approach is reminiscent of, and may well have inspired, Spinoza's treatment of the miracle of Christ's resurrection, which he addresses at length in a letter to Henry Oldenburg.

\begin{displayquote}
``For the rest, I accept Christ's passion, death, and burial literally, as you do, but His resurrection I understand allegorically. I admit, that it is related by the Evangelists in such detail, that we \emph{cannot deny that they themselves believed} Christ's body to have risen from the dead and ascended to heaven, in order to sit at the right hand of God, or that they believed that Christ might have been seen by unbelievers, if they had happened to be at hand, in the places where He appeared to His disciples; but in these matters they might, \emph{without injury to Gospel teaching}, have been \emph{deceived, as was the case with other prophets} mentioned in my last letter. But Paul, to whom Christ afterwards appeared, rejoices, that he knew Christ not after the flesh, but after the spirit.'' \\ \phantom{a} \hfill (Letter LXXVIII, emphasis mine)
\end{displayquote}

For Maimonides, the whole issue rested on the question of whether the Universe is eternal. The doctrine of the eternity of the universe has a curious history. It appears to have been held by Aristotle and the Peripatetic School  --- perhaps scandalously so in the face of the creation accounts of all three Testaments --- as a demonstrable truth, which, even on Maimonides' own account, would be sufficient grounds to render the literal meaning of the opening of Genesis moot. Maimonides wrestled long and hard with this theory, and in the final analysis appears convinced that it is only \emph{possibly} true, and not necessarily so. %It is important to note that this opinion --- that the eternity of the Universe cannot be definitively demonstrated by the prevailing science of his day --- was a \emph{scientific} one, not a religious one. The authors whose work he uses to come to this determination are Peripatitic ones.
\begin{displayquote}
``But the Eternity of the Universe has not been proved; a mere argument in favour of a certain theory is not sufficient reason for rejecting the literal meaning of a Biblical text, and explaining it figuratively, when the opposite theory can be supported by an equally good argument.'' \hfill (GP II.25)
\end{displayquote}

But then, so is the revelation at Sinai --- it is \emph{possibly} true under the hypothesis that the Universe had a beginning. 
\begin{displayquote}
``Accepting the Creation, we find that miracles are possible, that Revelation is possible, and that every difficulty in this question is removed.'' \hfill (GP II.25)
\end{displayquote}

Whether or not one believes that the Universe is eternal is, for Maimonides, \emph{scientifically} undemonstrable. It is, however, necessary to believe in the premise that the Universe had a beginning if one is to preserve the \emph{possibility of religious belief}. Apparently, however, it is an open question whether this premise should be treated as an axiom or as a statement that's provable in itself. 
\begin{displayquote}
``All who follow the Law of Moses, our Teacher, and Abraham, our Father, and all who adopt similar theories, assume that nothing is eternal except God, and that the theory of \emph{creatio ex nihilo} includes nothing that is impossible, whilst some thinkers even regard it as an established truth." \hfill (GP II.13)
\end{displayquote}

He does not claim to have found a rationally-grounded basis for believing that the Universe had a beginning --- even less does he find a Scriptural basis for such a belief --- but he admits that miracles and revelation are only possible under the hypothesis that the Universe had a beginning.

One might surmise that Maimonides' reasoning went like this. There are two hypotheses: A. the universe is eternal, and B. the universe had a beginning. A contradicts scripture while B has support from scripture; therefore, he chooses to believe in B.

In fact, his reasoning was more like this. There are two hypotheses: A. the universe is eternal, and B. the universe had a beginning. If we accept A, then we must admit that the world is governed by Avicennian necessitarianism, and in such a world miracles and revelation are not possible at all. If we accept B, then in such a world \emph{it is possible} for miracles and revelation to happen. Therefore, he chooses to believe in B.

For Maimonides, the choice was between two, mutually exclusive, equally non-demonstrable but reasonably plausible hypotheses; the one leading to the strict necessitarianism of Avicenna and the other leading to the \emph{possibility of} Law with a capital `L'. In the face of this choice, he appears to make a leap of faith --- grounded, however, not in an appeal to mystery but in a sort of Occam's Razor. The argument goes something like this.

Under Aristotle's theory, it was very difficult to explain the apparent paths of the planets in the sky. Epicycles and deferrents were a solution, but --- as long recognized by the astronomers of Andalusia --- hardly an elegant one. Maimonides lived in a scientifically advanced age, at a time when he could confidently state that ``the science of Astronomy was not yet fully developed, and ... in the days of Aristotle the motions of the spheres were not known so well as they are at present.''. In this scientific atmosphere, whose latent contradictions would in a few more centuries give birth to the Copernican Revolution, Maimonides believed that Aristotle's system of the world just did not offer a compelling explanation of all observed astronomical phenomena. We could perhaps even say that the pyrotechnics needed to sustain Aristotle's system verged on the nonsensical/irrational. On the other hand, Moses' explanation provided a quite satisfactory, even rational solution: God willed it to be so that Mercury sometimes goes retrograde.

Maimonides suggests that a world that is fully explained by `Science' is a world that has no room for the God of Abraham, Isaac and Jacob, however much it may commend the God of Aristotle or Avicenna (or Spinoza). 
\begin{displayquote}
``Everything is, according to him [i.e., Aristotle], the result of a law of Nature, and not the result of the design of a being that designs as it likes, or the determination of a being that determines as it pleases. He has not carried out the idea consistently, and it will never be done.'' \hfill (GP II.19)
\end{displayquote}

This is where Maimonides' God veers into a `god of the gaps'. The ominous and decided tone of `it will never be done' reflects at once Maimonides' reverence for the mystery of an Infinite God whose works can never be fully comprehended by the finite intellect of Man, and the modern sensibility that all scientific theories are `works in progress', and that our explanations for natural phenomena are only adequate/useful representations.

But \emph{what if} the idea could be carried out consistently? What would Maimonides think if the irrationalities inherent in the best scientific explanations of the day were resolved, and thereby a sound proof for the eternity of the world available? We need not speculate on contrafactuals here, for the Rambam has told us directly:
\begin{displayquote}
``If the Creation had been demonstrated by proof ... all arguments of the philosophers against us would be of no avail. If, on the other hand, Aristotle had a proof for his theory, \emph{the Law in its entirety would be rejected}, and \emph{we should be forced to other opinions}'' \hfill (GP II.25, emphasis added)
\end{displayquote}


\section*{Alternatives}
In this paper, I argue for two parallel theses, both of which can be strongly supported using textual evidence from within \GP.
\begin{enumerate}
\item The first regards a hypothetical situation.
\begin{enumerate}
\item Maimonides wrote of a hypothetical situation in which the world could fully be explained by the laws of nature, and
\item he held that, under such circumstances, belief in the Law could not be rationally defensible.
\end{enumerate}
\item The second regards the prevailing situation in his time.
\begin{enumerate}
\item Maimonides had strong reason to believe that the world as he knew it could not fully be explained by the laws of nature, and
\item he believed that, under the prevailing circumstances, belief in the Law was rationally defensible.
\end{enumerate}
\end{enumerate}

For each thesis, we can add some further corollaries which are our extrapolations from the same.

\subsection*{1(a)}
What I call here the `hypothetical situation in which the world could fully be explained by the laws of nature' was in Maimonides' time the prevailing belief among Aristotelian philosophers. He succinctly summarizes this position thus,
\begin{displayquote}
according to Aristotle, and according to all that defend his theory, the Universe is inseparable from God; He is the cause, and the Universe the effect; and this effect is a necessary one... For it is necessary that the whole, the cause as well as the effect, exist in this particular manner, it is impossible for them not to exist, or to be different from what they actually are. This leads to the conclusion that the nature of everything remains constant, that nothing changes its nature in any way, and that such a change is impossible in any existing thing. \hfill (GP II.19)
\end{displayquote}


Much of the confusion engendered by \GP has its roots in the space given in the Guide to this \emph{hypothetical} situation, which Maimonides did not, in fact, believe to be the case. He addresses this situation at length in the Guide in pursuit of the `fifth reason' for contradictions mentioned in the Introduction (preface?) to the first part, where Maimonides states:
\begin{displayquote}
``The fifth cause is traceable to the use of a certain method adopted in teaching and expounding profound problems. Namely, a difficult and obscure theorem must sometimes be mentioned and \emph{assumed as known}, for the illustration of some elementary and intelligible subject which must be taught beforehand the commencement being always made with the easier thing. The teacher must therefore facilitate, in any manner which he can devise, the explanation of those theorems, which \emph{have to be assumed as known}, and he must content himself with giving a general though \emph{somewhat inaccurate} notion on the subject. It is, for the present, explained according to the capacity of the students, that they may comprehend it as far as they are required to understand the subject. Later on, the same subject is thoroughly treated and fully developed in its right place.''\\ \phantom{a} \hfill (GP I Introd., emphasis mine)
\end{displayquote}

\newpage

\section*{On God as a Free Cause}

In GP II.22, Maimonides raises doubts about Aristotelian physics and points out the inability of the science of his day to explain all observed phenomena. He says that of the two hypotheses: (1) Aristotle's hypothesis that the universe is eternal and that nature does not change, and (2) the Mosaic hypothesis that the universe was created in time as an act of will by God, (1) fails to explain everything satisfactorily, but (2) offers a possible answer to questions that (1) does not satisfactorily answer. In the course of this explanation, Maimonides offers some illuminating remarks about what it means for God to be a `free' cause.

After enumerating some of the astronomical difficulties with Aristotle's physics, he says
\begin{displayquote}
If we, however, assume design and determination of a Creator, in accordance with His incomprehensible wisdom, all these difficulties disappear. [These difficulties] \emph{must} arise when we consider the whole Universe, not as the result of free will, but as the result of fixed laws of Nature: a theory which, on the one hand, is not in harmony with the existing order of things, and does not offer for it a sufficient reason or argument; and, on the other hand, implies many and great improbabilities. For, according to this [Aristotelian] theory God ... is in such relation to the Universe that He cannot change anything; if He wished to make the wing of a fly longer, or to reduce the number of the legs of a worm by one, He could not accomplish it. \hfill (GP II.22)
\end{displayquote}

Maimonides was quite uncomfortable with such a clockwork God, and considered such a God to be incompatible with the Jewish religion. For Maimonides, one of the perfections of God is that he is a `free cause'. 

On the other hand, Spinoza's God too is a `free cause', but what he means by this is radically different from the Rambam's understanding of this concept. We could even say he turned the Rambam's conception of God on its head, in the following excerpts:

\begin{displayquote}
God acts solely by the laws of his own nature, and is not constrained by anyone.\\  \phantom{a} \hfill Ethics I.17 \\
Others think that God is a free cause, because he can, as they think, bring it about, that those things which we have said follow from his nature—that is, which are in his power, should not come to pass, or should not be produced by him. But this is the same as if they said, that God could bring it about, that it should follow from the nature of a triangle that its three interior angles should not be equal to two right angles...  \hfill Ethics 1.17 Note
\end{displayquote}


\end{document}  







